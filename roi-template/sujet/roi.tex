% !TEX TS-program = pdflatex   
\documentclass[runningheads,12pt]{article} 
\usepackage{textcomp,color}
\def\headerTitle{CSE4312F12 Project -- ROI Solution}
\usepackage{sty/bsymb} %% Event-B symbols
\usepackage{sty/eventB} %% REQ and ENV
%%UML
%UML
\usepackage{tikz}
%\usepackage[school,simplified]{pgf-umlcd}
%\usepackage[simplified]{sty/pgf-umlcd}
\usepackage{mdframed}%% put frame around a table
\usepackage{rotating} 
\usepackage{subfigure}
\usepackage{sty/CalcStyleV8}

\def\newlineindent{\newline\indent}
\def\require{\quad\textbf{require}~}           
\def\ensure{\quad\textbf{ensure}~}
\def\param{\quad\textbf{param}~}
\def\where{\quad\textbf{where}~}
\def\when{\quad\textbf{when}~}
\def\axiom{\textbf{axiom}~}
\def\comment{\quad \texttt{--}}
\def\ddo{\quad\textbf{do}~} 
\def\any{\quad\textbf{any}~}   
\def\cset{\textbf{carrier set}~}
\def\var{\textbf{var}~}
\def\const{\textbf{const}~}
\def\is{~\textbf{is}~}
\def\suchthat{\,|\,}
\def\yields{\,\bullet\,}
\def\itholds{\,\bullet\,}
\def\inv{\textbf{inv}~}
\def\eqdef{~\triangleq~}
\def\event{\textbf{event}~}
\def\proc{\textbf{proc}~}
\def\query{\textbf{query}~}
\def\init{\textbf{init}~}
\def\use{\textbf{use}~}
\def\constraint{\textbf{constraint}~}
\def\property{\textbf{property}~}
\def\regexp{REGEXP}
\def\infix{\textbf{infix }}
\def\prefix{\textbf{prefix }}
\def\dummy{\textbf{dummy }}
\def\theorem{\textbf{theorem }}
\def\squote#1{\textrm{\textquotesingle#1\textquotesingle}}

\usepackage{listings,xcolor}%%lstlang0.sty
\definecolor{codebg}{rgb}{0.92,0.92,0.92}      

\usepackage{multirow}    
\usepackage{pbox}

% block in math mode
\def\block{\array{@{}l@{}}}
\let\endblock=\endarray

% Inline Eiffel Code
\newcommand{\e}[1]{\lstinline[language=OOSC2Eiffel]|#1|}


% Eiffel figure setup
\lstnewenvironment
  {ecodel}[1][]
  {\lstset{language=OOSC2Eiffel,float=htb,columns=fullflexible,backgroundcolor=\color{codebg},captionpos=b,
  xleftmargin=1cm,numberfirstline=true,numberstyle=\hspace*{1cm},numbers=left,mathescape=true,#1}}
  {}

\lstnewenvironment
  {ecodel2}[1][]
  {\lstset{language=OOSC2Eiffel,float=htb,columns=fullflexible,backgroundcolor=\color{codebg},captionpos=t,
  xleftmargin=1cm,mathescape=true,#1}}
  {}
                          
%%%%% Pretty print an Eiffel feature
%%%%% Arguments:
%%%%% 1. Name of the routine, parameters and return type if applicable.
%%%%% 2. Comments of the routine.
%%%%% 3. Pre-condition (can be split into multiple lines; math symbols must be enclosed within $...$)
%%%%% 4. Implemnetation (it's recommanded that use \texttt{...} to wrap each line to emphasize that it's code, not spec.)
%%%%% 5. Post-condition (can be split into multiple lines; math symbols must be enclosed within $...$)
%%%%%
%%%%% Usage: \feature{f (params) : return }{comment}{pre}{do}{post}  
%%%%%       will produce the following
%%%%%
%%%%%       f (params) : return
%%%%%       		-- comment
%%%%%       	require pre
%%%%%          	do
%%%%%				imp
%%%%%			ensure post
%%%%%			end        
%%%%% Special case:
%%%%% - Contract view --> \feature{f (params) : return }{comment}{pre}{}{post} 
%%%%% - Precondition only -->  \feature{f (params) : return }{comment}{pre}{}{} 
%%%%% - Postcondition only --> \feature{f (params) : return }{comment}{}{}{post}      
%%%%% - Implementation only --> \feature{f (params) : return }{comment}{}{imp}{} 

\def\requires{\textbf{require}}           
\def\ensures{\textbf{ensure}}    
\def\Do{\textbf{do}}
\def\End{\textbf{end}}   
                                        
\newlength{\requirewidth}
\settowidth{\requirewidth}{require}  
\def\old{\mathop{\textbf{old}}}   

\newcommand\contract[2]{  
	\begin{array}{@{\quad}l@{\quad}l}
	\begin{minipage}[t]{\widthofpbox{#1}} #1 \end{minipage} 
	&
	\begin{minipage}[t]{\widthofpbox{#2}} #2 \end{minipage}
	\end{array}}             

\newcommand{\feature}[5]{
  \def\comments{#2}
  \def\pre{#3}
  \def\imp{#4}
  \def\post{#5}     
  \begin{array}[!htbp]{|@{\quad}l@{\quad}|}
    \hline
    \\[-1ex]
    #1 \\   

	\ifx\comments\empty
	\else
	\quad\quad\text{-}\text{- }\text{\comments} \\
	\fi   
	
	\ifx\pre\empty
	\else                        
		\contract{\requires}{\pre} \\
	\fi  
	
	\ifx\imp\empty
	\else
	\begin{array}[!htbp]{@{\quad}l}
	    \Do \\ 
	    \begin{array}[!htbp]{@{\quad}l}
			\imp
		\end{array}
	\end{array}\\
	\fi        
	
	\ifx\post\empty
	\else
	\contract{\ensures}{\post} \\
	\fi 
	  
	\ifx\imp\empty 
	% contract view
    \else
     \quad\Bend
    \fi \\[1ex] 
    \hline
  \end{array}
}

\newcommand{\module}
{
 \begin{array}[!htbp]{|@{\quad}l@{\quad}|}
    \hline
   
    \hline
  \end{array}
}

\usepackage{amssymb,amsmath}
\setcounter{tocdepth}{3}
\usepackage{graphicx}
%\usepackage[text={16cm,23cm}]{geometry} %% use more of the page
\usepackage{url} 

\usepackage{enumitem} 
\usepackage{float,afterpage}
\usepackage{caption}

%reduce spacing in lists
\usepackage[section]{placeins}
\setlist{nolistsep,leftmargin=*} 

\usepackage{fancyhdr,lastpage}
\lhead{\rm \headerTitle}
\rhead {\rm Page \thepage~of \pageref{LastPage}}
\lfoot{}\cfoot{}\rfoot{}
\pagestyle{fancy}

%%%%%%%%%Definitions%%%%%%%%%%

\def\implies{\Rightarrow } 
\def\Fl{\mathbb{F}}%%FLOAT
\def\Bl{\mathbb{B}}%%BOOLEAN
\def\St{\mathbb{S}}%%STRING
%%%%%%%%%%%%%%%%%%%%%%%%%%%%

%For reducing size of section headings
\usepackage{sectsty}
\sectionfont{\normalsize}
\subsectionfont{\small}
\subsubsectionfont{\small}
\usepackage{titlesec}
\titlespacing*{\section}{0pt}{*1.2}{*0.5}
\titlespacing*{\subsection}{0pt}{*0.5}{*0.5}
\titlespacing*{\subsubsection}{0pt}{*0.5}{*0.5}

%http://www.eng.cam.ac.uk/help/tpl/textprocessing/squeeze.html
%By default, LaTeX doesn't like to fill more than 0.7 of a text page with tables and graphics, nor does it like too many figures per page. This behaviour can be changed by placing lines like the following before \begin{document}
\renewcommand\floatpagefraction{.9}
\renewcommand\topfraction{.9}
\renewcommand\bottomfraction{.9}
\renewcommand\textfraction{.1}   
\setcounter{totalnumber}{50}
\setcounter{topnumber}{50}
\setcounter{bottomnumber}{50}

%\usepackage{CalcStyleV8}
%\usepackage{subfigure}


\usepackage{sty/tikz-uml}%% Put this last
\usepackage{lscape}
 


\begin{document}

\title{CSE4312F12 Project Solution\\
ROI}

\author{Damien Gruel (cse23089@cse.yorku.ca)
\and Ludovic Lavalette (cse23088@cse.yorku.ca)}

\date{\today}

\maketitle

\section*{Note}
\begin{itemize}
\item A customer elicitation session was held during class on Tuesday November 6, 2012. If you were not there sure to catch up with a fellow student who was there.
\item This template is handed out \emph{caveat emptor}.  There may be errors and wrong information. It is ultimately your responsibility to elicit the correct requirements from the customer and to ensure that you satisfy the customer goals and specify correct output from the input.
\item Your are required to correct any errors or ambiguities in this template and use this template to produce your final requirements document.
\end{itemize}

\section*{Revisions}

\begin{tabular}{|l|l|p{3in}|}
\hline
Date & Revision& Description \\ 
\hline
10 October  2012 
& 1.0       
& Initial customer elicitation\\ 
\hline
15 November 2012
& 2.0       
& Initial Student solution       \\
\hline
1 December 2012
& 3.0       
& Final Student solution       \\
\hline
\end{tabular}

\newpage
\tableofcontents
\listoffigures
\listoftables

\newpage

\section{Elicitation of customer goals}
Our customers are the CEO and IT manager of Investment Corp. They desire an easy-to-use application to keep track of their return on their investments (ROI). A requirements elicitation session with our customers yielded the following issues and goals:

Each month, the customer receives a portfolio statement. The customer is not interested in keeping track of individual stocks, bonds etc. in their portfolio. What they would like to know is how the fund is doing thus far, i.e. over all data as well as over a specified evaluation period. There are free applications such as GnuCash for calculating ROI, but such applications require too much input information and are too complex for what is required. Customers also want to compare their return on their investment portfolios with respect to standard benchmarks \cite{gips,Lawton09}. Benchmark data (when available) is either at year-end or year-to-date.

Customers might have multiple portfolios e.g. one for themselves and one for their spouse. All returns on investments are expressed in percentage per annun and all calculations must be done to industry standards. Customers might enter deposits into the investment account at arbitrary dates during the year. They may also withdraw money at arbitrary dates. This will affect the ROI calculation.

The input data for each portfolio is maintained by the customer as a CSV (comma separated value) text file as in Fig.~\ref{fig:csv}. This allows them to keep track of their data on their smart phones or other devices. On the Windows desktop, double clicking on the file opens in Excel as in in the figure. The input also reports agent fees and, possibly a benchmark. For market values, cash flow (positive in, negative out) and agent fees, where no value is shown the default is zero.

Customers receive statements (sometimes monthly or every couple of months and always at the end of each year – Dec. 31) from their investment brokers. The statement has a bottom line viz. the total value of all their investments to date (which includes bonds, stocks, etc.). Customers enter the total value of their portfolio at that date. For example, customers might receive a statement dated December 31, 2006 for \$10,000. This is the value of the portfolio at the end of December 31, 2007, and is entered as \$10,000 dated January 1, 2007. All entries thus reflect the value of the portfolio at the beginning of the day with deposits and withdrawals occurring during the rest of the day. 

\begin{figure}
\centering
\includegraphics[scale=0.6]{inputs/excel-csv.pdf}
% !TEX TS-program = pdflatex   
\documentclass[runningheads,12pt]{article} 
\usepackage{textcomp,color,longtable,landscape}
\def\headerTitle{CSE4312F12 Project -- ROI Solution}
\usepackage{sty/bsymb} %% Event-B symbols
\usepackage{sty/eventB} %% REQ and ENV
%%UML
%UML
\usepackage{tikz}
%\usepackage[school,simplified]{pgf-umlcd}
%\usepackage[simplified]{sty/pgf-umlcd}
\usepackage{mdframed}%% put frame around a table
\usepackage{rotating} 
\usepackage{subfigure}
\usepackage{sty/CalcStyleV8}

\def\newlineindent{\newline\indent}
\def\require{\quad\textbf{require}~}           
\def\ensure{\quad\textbf{ensure}~}
\def\param{\quad\textbf{param}~}
\def\where{\quad\textbf{where}~}
\def\when{\quad\textbf{when}~}
\def\axiom{\textbf{axiom}~}
\def\comment{\quad \texttt{--}}
\def\ddo{\quad\textbf{do}~} 
\def\any{\quad\textbf{any}~}   
\def\cset{\textbf{carrier set}~}
\def\var{\textbf{var}~}
\def\const{\textbf{const}~}
\def\is{~\textbf{is}~}
\def\suchthat{\,|\,}
\def\yields{\,\bullet\,}
\def\itholds{\,\bullet\,}
\def\inv{\textbf{inv}~}
\def\eqdef{~\triangleq~}
\def\event{\textbf{event}~}
\def\proc{\textbf{proc}~}
\def\query{\textbf{query}~}
\def\init{\textbf{init}~}
\def\use{\textbf{use}~}
\def\constraint{\textbf{constraint}~}
\def\property{\textbf{property}~}
\def\regexp{REGEXP}
\def\infix{\textbf{infix }}
\def\prefix{\textbf{prefix }}
\def\dummy{\textbf{dummy }}
\def\theorem{\textbf{theorem }}
\def\squote#1{\textrm{\textquotesingle#1\textquotesingle}}

\usepackage{listings,xcolor}%%lstlang0.sty
\definecolor{codebg}{rgb}{0.92,0.92,0.92}      

\usepackage{multirow}    
\usepackage{pbox}

% block in math mode
\def\block{\array{@{}l@{}}}
\let\endblock=\endarray

% Inline Eiffel Code
\newcommand{\e}[1]{\lstinline[language=OOSC2Eiffel]|#1|}


% Eiffel figure setup
\lstnewenvironment
  {ecodel}[1][]
  {\lstset{language=OOSC2Eiffel,float=htb,columns=fullflexible,backgroundcolor=\color{codebg},captionpos=b,
  xleftmargin=1cm,numberfirstline=true,numberstyle=\hspace*{1cm},numbers=left,mathescape=true,#1}}
  {}

\lstnewenvironment
  {ecodel2}[1][]
  {\lstset{language=OOSC2Eiffel,float=htb,columns=fullflexible,backgroundcolor=\color{codebg},captionpos=t,
  xleftmargin=1cm,mathescape=true,#1}}
  {}
                          
%%%%% Pretty print an Eiffel feature
%%%%% Arguments:
%%%%% 1. Name of the routine, parameters and return type if applicable.
%%%%% 2. Comments of the routine.
%%%%% 3. Pre-condition (can be split into multiple lines; math symbols must be enclosed within $...$)
%%%%% 4. Implemnetation (it's recommanded that use \texttt{...} to wrap each line to emphasize that it's code, not spec.)
%%%%% 5. Post-condition (can be split into multiple lines; math symbols must be enclosed within $...$)
%%%%%
%%%%% Usage: \feature{f (params) : return }{comment}{pre}{do}{post}  
%%%%%       will produce the following
%%%%%
%%%%%       f (params) : return
%%%%%       		-- comment
%%%%%       	require pre
%%%%%          	do
%%%%%				imp
%%%%%			ensure post
%%%%%			end        
%%%%% Special case:
%%%%% - Contract view --> \feature{f (params) : return }{comment}{pre}{}{post} 
%%%%% - Precondition only -->  \feature{f (params) : return }{comment}{pre}{}{} 
%%%%% - Postcondition only --> \feature{f (params) : return }{comment}{}{}{post}      
%%%%% - Implementation only --> \feature{f (params) : return }{comment}{}{imp}{} 

\def\requires{\textbf{require}}           
\def\ensures{\textbf{ensure}}    
\def\Do{\textbf{do}}
\def\End{\textbf{end}}   
                                        
\newlength{\requirewidth}
\settowidth{\requirewidth}{require}  
\def\old{\mathop{\textbf{old}}}   

\newcommand\contract[2]{  
	\begin{array}{@{\quad}l@{\quad}l}
	\begin{minipage}[t]{\widthofpbox{#1}} #1 \end{minipage} 
	&
	\begin{minipage}[t]{\widthofpbox{#2}} #2 \end{minipage}
	\end{array}}             

\newcommand{\feature}[5]{
  \def\comments{#2}
  \def\pre{#3}
  \def\imp{#4}
  \def\post{#5}     
  \begin{array}[!htbp]{|@{\quad}l@{\quad}|}
    \hline
    \\[-1ex]
    #1 \\   

	\ifx\comments\empty
	\else
	\quad\quad\text{-}\text{- }\text{\comments} \\
	\fi   
	
	\ifx\pre\empty
	\else                        
		\contract{\requires}{\pre} \\
	\fi  
	
	\ifx\imp\empty
	\else
	\begin{array}[!htbp]{@{\quad}l}
	    \Do \\ 
	    \begin{array}[!htbp]{@{\quad}l}
			\imp
		\end{array}
	\end{array}\\
	\fi        
	
	\ifx\post\empty
	\else
	\contract{\ensures}{\post} \\
	\fi 
	  
	\ifx\imp\empty 
	% contract view
    \else
     \quad\Bend
    \fi \\[1ex] 
    \hline
  \end{array}
}

\newcommand{\module}
{
 \begin{array}[!htbp]{|@{\quad}l@{\quad}|}
    \hline
   
    \hline
  \end{array}
}

\usepackage{amssymb,amsmath}
\setcounter{tocdepth}{3}
\usepackage{graphicx}
%\usepackage[text={16cm,23cm}]{geometry} %% use more of the page
\usepackage{url} 

\usepackage{enumitem} 
\usepackage{float,afterpage}
\usepackage{caption}

%reduce spacing in lists
\usepackage[section]{placeins}
\setlist{nolistsep,leftmargin=*} 

\usepackage{fancyhdr,lastpage}
\lhead{\rm \headerTitle}
\rhead {\rm Page \thepage~of \pageref{LastPage}}
\lfoot{}\cfoot{}\rfoot{}
\pagestyle{fancy}

%%%%%%%%%Definitions%%%%%%%%%%

\def\implies{\Rightarrow } 
\def\Fl{\mathbb{F}}%%FLOAT
\def\Bl{\mathbb{B}}%%BOOLEAN
\def\St{\mathbb{S}}%%STRING
%%%%%%%%%%%%%%%%%%%%%%%%%%%%

%For reducing size of section headings
\usepackage{sectsty}
\sectionfont{\normalsize}
\subsectionfont{\small}
\subsubsectionfont{\small}
\usepackage{titlesec}
\titlespacing*{\section}{0pt}{*1.2}{*0.5}
\titlespacing*{\subsection}{0pt}{*0.5}{*0.5}
\titlespacing*{\subsubsection}{0pt}{*0.5}{*0.5}

%http://www.eng.cam.ac.uk/help/tpl/textprocessing/squeeze.html
%By default, LaTeX doesn't like to fill more than 0.7 of a text page with tables and graphics, nor does it like too many figures per page. This behaviour can be changed by placing lines like the following before \begin{document}
\renewcommand\floatpagefraction{.9}
\renewcommand\topfraction{.9}
\renewcommand\bottomfraction{.9}
\renewcommand\textfraction{.1}   
\setcounter{totalnumber}{50}
\setcounter{topnumber}{50}
\setcounter{bottomnumber}{50}

%\usepackage{CalcStyleV8}
%\usepackage{subfigure}


\usepackage{sty/tikz-uml}%% Put this last
\usepackage{lscape}
 


\begin{document}

\title{CSE4312F12 Project Solution\\
ROI}

\author{Damien Gruel (cse23089@cse.yorku.ca)
\and Ludovic Lavalette (cse23088@cse.yorku.ca)}

\date{\today}

\maketitle

\section*{Note}
\begin{itemize}
\item A customer elicitation session was held during class on Tuesday November 6, 2012. If you were not there sure to catch up with a fellow student who was there.
\item This template is handed out \emph{caveat emptor}.  There may be errors and wrong information. It is ultimately your responsibility to elicit the correct requirements from the customer and to ensure that you satisfy the customer goals and specify correct output from the input.
\item Your are required to correct any errors or ambiguities in this template and use this template to produce your final requirements document.
\end{itemize}

\section*{Revisions}

\begin{tabular}{|l|l|p{3in}|}
\hline
Date & Revision& Description \\ 
\hline
10 October  2012 
& 1.0       
& Initial customer elicitation\\ 
\hline
15 November 2012
& 2.0       
& Initial Student solution       \\
\hline
1 December 2012
& 3.0       
& Final Student solution       \\
\hline
\end{tabular}

\newpage
\tableofcontents
\listoffigures
\listoftables

\newpage

\section{Context Diagram}



The following diagram is the context diagram for the ROI system.\\
The only monitored variable is the CSV file (provided by the user), which contains the evaluation dates (\textit{start} and \textit{end}).\\
\\
The format of the output is the following (whole input = everything between the earliest date and the latest date in the sequence of tuple data):\\
-------------------------------------------\\
Whole input: yyyy-mm-dd to yyyy-mm-dd\\
TWR: ?? \%\\
ROI: ?? \%\\
Benchmark: ?? \%\\
--------------------------------------------\\
-------------------------------------------\\
Evaluation Period: yyyy-mm-dd to yyyy-mm-dd\\
TWR: ?? \%\\
ROI: ?? \%\\
Benchmark: ?? \%\\
--------------------------------------------\\
\\
\\
The controlled variables are also a warning (if a calculation is not possible) and an error (if the CSV file is not valid).\\

\begin{figure}
\centering
\includegraphics[scale=1.0]{inputs/context.pdf}
\caption{Context diagram for the ROI system}
\label{fig:ROI-context}
\end{figure}
\section{Dictionary}

\smallskip


\noindent\textbf{Agent fees}: Money that the customer pays to the investment advisor to run the account.\smallskip

\noindent\textbf{Benchmark}: Standard used as a point of reference for evaluating performance.\smallskip

\noindent\textbf{Cash Flow}: Revenue or expense stream that changes a cash account over a given period.\smallskip

\noindent\textbf{CSV}: Comma Separated Value file format used to store tabular data in which numbers and text are stored in plain-text form that can be easily written and read in a text editor.\smallskip

\noindent\textbf{Customer}: The user of the software system.\smallskip

\noindent\textbf{Evaluation Period}: a start and end date (provided by the user) for the portfolio history over which the return on investment is calculated.\smallskip

\noindent\textbf{GIPS}: Global Investment Performance Standards\smallskip

\noindent\textbf{Investment broker}: Runs the portfolio on behalf of the customer and supplies portfolio accounts.\smallskip

\noindent\textbf{Portfolio statement}: List of all investments and current value.\smallskip

\noindent\textbf{Portfolio History}: the historical data of investment performance over time that the customer stores about their investments as gleaned from their monthly or yearly investment accounts. Usually stored by customers in a CSV file (see Figure 1).\smallskip

\noindent\textbf{ROI}: Return On Investment: Performance measure used to evaluate the efficiency of an investment.\smallskip

\noindent\textbf{TWR}: Time Weighted Return: Measure of the compound rate of growth in a portfolio.\smallskip

\noindent\textbf{Tuple data}: \textit{date}, \textit{market value}, \textit{cash flow}, \textit{agent fees} and \textit{benchmark}.\smallskip

%%%%%%%%%%%%%%%%%%%%%%%%%%%%%%%%%%%
\section{E/R-descriptions}

\subsection{E-descriptions}

{\centering
\begin{tabular}{|l|p{9cm}|p{5cm}|}
\hline
\textbf{ID} & \textbf{Description} & \textbf{Comment}\\

\hline
E1 & Customers create and store a portfolio history, i.e. the historical data of their investment performance as determined from portfolio statements. & \\

\hline
E2 & Customers store their portfolio history as a CSV text file. CSV files may be prepared on editors of any operating system and encoded as ANSI or UTF-8. & \\

\hline
E3 & Every portfolio history has a name. & \\

\hline
E4 & Optionally, every portfolio history has a description, account number, email, address, and phone number fields. & \\

\hline
E5 & A portfolio history records investment performance in a non-empty sequence of tuple data, each tuple having the fields: date, market value, cash flow, agent fees and benchmark. & See \textit{tr} of TWR\_ROI\_CALCULATION (Fig.~\ref{tab:twr_calculation})\\

\hline
E6 & When there is a customer contribution, the cash flow is a positive number. For a withdrawal, the number is negative. & \\

\hline
E7 & Agent fees can be internal (deducted from within the portfolio) or external (additional amounts paid by the customer to the investment broker). The portfolio history reflects only external agent fees, always reported as a non-negative amount. & \\

\hline
E8 & Optionally, every portfolio has an evaluation period that is between the start and end date of the historical performance data. & See Invariant 1 of TWR\_ROI\_CALCULATION (Fig.~\ref{tab:twr_calculation})\\
\hline
\end{tabular}
}

%%%%%%%%%%%%%%%%
\newpage
\subsection{R-descriptions}

{\centering
\begin{longtable}{|l|p{9cm}|p{5cm}|}
\hline
\textbf{ID} & \textbf{Description} & \textbf{Comment}\\

\hline
R1 & All return on investment calculations shall follow the GIPS standard. & See twr, roi, benchmark (Fig.~\ref{tab:twr_calculation}) \\

\hline
\end{longtable}
%%%
\centering
\begin{longtable}{|l|p{9cm}|p{5cm}|}
\hline
\multicolumn{3}{|c|}{\textbf{Evaluation period}} \\

\hline
R2.1 & The evaluation period is in range. & See Invariant 1 of TWR\_ROI\_CALCULATION (Fig.~\ref{tab:twr_calculation})\\

\hline
R2.2 & If no evaluation period is provided, then the start date is the earliest date and the end date the latest date in the sequence of tuple data. & See Start\_Valid and End\_Valid in Function Table\\

\hline
R2.3 & If the evaluation dates are not valid, then the following error message shall be displayed to the user: "Invalid\_Evaluation\_Period" & See Function Table \\

\hline
\end{longtable}
%%%
\centering
\begin{longtable}{|l|p{9cm}|p{5cm}|}

\hline
\multicolumn{3}{|c|}{\textbf{CSV file}} \\

\hline
R3.1 & Every data tuple (row in the CSV file) has a date and a non-negative market value. & See Invariant 2 of TWR\_ROI\_CALCULATION (Fig.~\ref{tab:twr_calculation})\\

\hline
R3.2 &  Dates in the tuples are unique and ordered. & See Invariant 3 of TWR\_ROI\_CALCULATION (Fig.~\ref{tab:twr_calculation})\\

\hline
R3.3 & No withdrawal in the tuple data can be greater than the market value. & See Invariant 4 of TWR\_ROI\_CALCULATION (Fig.~\ref{tab:twr_calculation})\\

\hline
R3.4 & An account cannot grow from zero market value and cash flow. & See Invariant 5 of TWR\_ROI\_CALCULATION (Fig.~\ref{tab:twr_calculation})\\

\hline
R3.5 &  For each tuple, the market value plus cash-flow plus agent-fees must be non-zero. & See precondition 3 of feature \textit{twr} of TWR\_ROI\_CALCULATION (Fig.~\ref{tab:twr_calculation})\\

\hline
R3.6 &  Error message: If the CSV file is not valid (i.e. if any of the conditions mentioned above do not hold), then the following error message shall be displayed to the user: "Invalid\_file". & See Function table\\

\hline
R3.7 &  All dates must be in ISO format (yyyy-mm-dd). & \\

\hline
\end{longtable}
%%%
\centering
\begin{longtable}{|l|p{9cm}|p{5cm}|}

\hline
\multicolumn{3}{|c|}{\textbf{Calculation of the TWR}} \\

\hline
R4.1 & The system should provide two TWR : one for the evaluation period, and one for the whole input. & See Function Table \\

\hline
R4.2 & If the evaluation period is less than a year, then the TWR shall be reported in absolute terms as a percentage return (i.e. it is not annualized). If the evaluation period is a year or more, then the TWR is annualized to a percentage per year. & See postcondition of  \textit{annual\_compounded\_TWR} of TWR\_ROI\_CALCULATION (Fig.~\ref{tab:twr_calculation})\\

\hline
R4.3 & The annualized TWR shall be reported as a percentage. & See \textit{annual\_compounded\_TWR} of TWR\_ROI\_CALCULATION (Fig.~\ref{tab:twr_calculation})\\

\hline
R4.4 & Agent fees are treated like a deposit. & See \textit{annual\_compounded\_TWR} of TWR\_ROI\_CALCULATION (Fig.~\ref{tab:twr_calculation})\\

\hline
R4.5 & Warning message: If the TWR is not calculable, then a warning message shall be displayed to the user. & See Function Table\\

\hline
\end{longtable}
%%%
\centering
\begin{longtable}{|l|p{9cm}|p{5cm}|}

\hline
\multicolumn{3}{|c|}{\textbf{Calculation of the ROI}} \\

\hline
R5.1 & The system should provide two ROI : one for the evaluation period, and one for the whole input. & See Function Table\\

\hline
R5.2 & The ROI shall be reported as a percentage. & See \textit{roi} of TWR\_ROI\_CALCULATION (Fig.~\ref{tab:twr_calculation})\\

\hline
R5.3 & Agent fees are treated like a deposit. & See \textit{roi} of TWR\_ROI\_CALCULATION (Fig.~\ref{tab:twr_calculation})\\

\hline
\end{longtable}
%%%
\centering
\begin{longtable}{|l|p{9cm}|p{5cm}|}

\hline
\multicolumn{3}{|c|}{\textbf{Calculation of the Benchmark}} \\

\hline
R6.1 & The system should provide two benchmarks : one for the evaluation period, and one for the whole input. & See Function Table\\

\hline
R6.2 & The benchmark shall be reported as a compounded ROI, if the benchmark figures are available for the evaluation period. & See \textit{benchmark} of TWR\_ROI\_CALCULATION (Fig.~\ref{tab:twr_calculation})\\

\hline
R6.3 & Warning message: If the benchmark is not calculable, then a warning message shall be displayed to the user. & See Function Table\\



\hline
\end{longtable}
}

\newpage

%%%%%%%%%%%%%%%%%%%%%%%%%%%%%%%%%%%
\section{Mathematical model}
%%%%%%%%%%%%%%%%%%%%%%%%%%%%%%%%%%%


\begin{figure}
\centering
\includegraphics[scale=0.8]{inputs/spec.pdf}
%\begin{figure}
\resizebox{\textwidth}{!}
{
\def\data{type DATA}
\def\input{type input-csv}
\def\roi{type roi}
\def\control{module control}
\renewcommand\tikzumlfillclasscolor{white}
\def\do{\quad\textbf{do}~} %%Needed
\centering
\resizebox{\textwidth}{!}{\begin{tikzpicture}

\umlclass{\data}
{
\use DATE\comment see birthday book\\
\cset $DATA \eqdef$\\
$\qquad  Data(date: DATE, mv:\Fl, cf:\Fl, af:\Fl,bench: \Fl\bunion\{Void\})$\\
$\qquad$\require $valid\_date(date)$\\

}
{
\comment the above declaration induces an injective function with accessors\\
$Data: DATE \cprod \Fl \cprod \Fl \cprod \Fl \cprod \Fl\bunion\{Void\} \tinj DATA$\\
\const $mv,cf,af : DATA \tfun \Fl$\\
\comment market-value, cash-flow and agent-fees\\
\const $bench: DATA \tfun \Fl\bunion\{Void\}$\comment benchmark\\
\axiom $\forall d:DATA \itholds d=Data(date(d),mv(d),cf(d),bench(d))$\\
\comment e.g. $DATA = \{ Data(\squote{2007-02-11},105000,15000,0,Void), \cdots\}$
}
        
\umlclass[x=-4.5, y=-6.4]{\input}
{
\use type $REGEXP, DATA$ \\
\textbf{param} $csvfile : \St$\\
\comment input parameter\\
\query $tr : SEQ[DATA]$\\
\comment sequence of transactions\\
\require $csvfile  \in VALID\_FILE$\\
$\query start, end: DATE$\\
\require $csvfile  \in VALID\_FILE$\\
}
{
\comment see Fig.~\ref{fig:input} for the details
}
        
\umlclass[x=4.5, y=-6.4]{\roi}
{
\use $input(cvsfile)$\\
\param $cvsfile:\St$\\
\query $report:\St$
}
{
\comment see function table for outputs\\
}
        
\umlclass[y=-12.4]{\control}
{
\comment specification pattern for transfomational programs
}
{
\event $roiReport \leftarrow invoke$\\
\any $file:\St$\\
$\quad$\use $roi(file)$ \textbf{as} $r$\\
\ddo $roiReport := r.report$\\\\
\comment dynamic use of module \textbf{roi}\\
\comment instantiate $cvsfile$ as $file$

}
        
\umlunicompo{\input}{\data}
\umlunicompo{\roi}{\data}
\umlunicompo{\roi}{\input}
\umlunicompo{\control}{\roi}
\end{tikzpicture}}
}
%\end{figure}
\caption{Module specification of return on investment}
\end{figure}

\begin{figure}
\resizebox{\textwidth}{!}{{\begin{tikzpicture}
\renewcommand\tikzumlfillclasscolor{white}
\umlclass{type input-csv}
{
\use type $REGEXP, DATA, DATE$ 
\comment we let $\epsilon = \{``"\}$, eol = $\{``\backslash\textrm{n}"\}$ etc.\\
\cset $DATA \eqdef Data(date: DATE, mv:\Fl, cf:\Fl, af:\Fl,bench: \Fl\bunion\{Void\})$\\
\textbf{param} $csvfile : \St$\comment input parameter\\
\query $tr : SEQ[DATA]$\comment sequence of transactions defined by axiom below\\
\require $csvfile  \in VALID\_FILE$\\
$\query start, end: DATE$\\
\require $csvfile  \in VALID\_FILE$\\
}
{
\const $VALID\_FILE : \regexp$\\
$\eqdef HEADER \cdot PARAMETERS \cdot \textrm{eol} \cdot ROW \cdot \textbf{*}(\textrm{eol} \cdot ROW)\cdot \textbf{*}(``,"|\textrm{eol})$\\
\const $HEADER: REGEXP$\\
$\eqdef \textbf{*} (HLINE \cdot eol) $\\
\const $HLINE:REGEXP$\\
$\eqdef \textbf{*} (\Sigma \backslash \textrm{eol}) \backslash (EV\_PER\cdot \textbf{*} \Sigma)$\\
\const $PARAMETERS:REGEXP$\\
$\eqdef EV\_PER \cdot DATE\_STR \cdot``\_ \textrm{to}\_" \cdot DATE\_STR \cdot \textbf{*}``,"\cdot \textrm{eol}\cdot COL\_HEAD$\\
\const $COL\_HEAD:REGEXP$\\
$ \eqdef +``," \cdot eol \cdot$\\ $\qquad``\textrm{Transaction\_Date,Market\_Value,Cash\_Flow,Agent\_Fees,Benchmark}"\cdot\textbf{*} ``,"$\\
\const $EV\_PER:REGEXP \eqdef ``\textrm{Evaluation\_Period:\_}" $\\
\const $ROW: \regexp$\\
$\eqdef (DATE\_STR \cdot ``,"\cdot  FLOAT\cdot ``,"  \cdot (FLOAT|\epsilon) \cdot ``," \cdot (FLOAT|\epsilon)\cdot$\\
$\qquad ``,"  \cdot (FLOAT\cdot``\%"|\epsilon)\cdot \textbf{*}``,")$\\
\const $s2d : DATE\_STR \tfun DATE$\comment see birthday book for $DATE$\\
\const $s2f: FLOAT \tfun \Fl$\comment deferred, $FLOAT$ is the string version of $\Fl$\\
\const $f2s : \Fl \tfun FLOAT$\comment deferred, see your favourite programming language\\
\const $d2s : DATE \tfun DATE\_STR$\comment deferred\\
\const $s2optf[G]: (FLOAT | \epsilon)\cprod G\tfun  \Fl \bunion G$\comment string-to-optional float \\
\where $\forall G \itholds s2optf \in (FLOAT | \epsilon)\cprod G\tfun  \Fl \bunion G$\\
\comment parameter $G$ is a set such as $\{Void\}$ or a default value such as $\{0\}$\\
\const $f : ROW \tfun DATA$\\
\textbf{dummy} $w:ROW$ and $s_0, s_1, s_2,s_3:\St$\\
\axiom 1:\comment definition of function $f$ that maps a row string to data\\
$\qquad\qquad w \in  (d2s(d)\cdot ``,"\cdot s_0 \cdot ``,"\cdot s_1 \cdot ``,"\cdot s_2 \cdot ``," \cdot s_3 \cdot \textbf{*}``,") $ \\ 
$\qquad ~~\land ~ (s_4 \cdot ``\%" = s_3 \lor s_4 = s_3 = \epsilon)$ \\
$\qquad \implies   f (w) = Data(d, s2f(s_0), s2optf(s_1,0), s2optf(s_2,0), s2optf(s_4,\text{Void}))$\\
\query $error : \Bl \eqdef textfile \not\in VALID\_FILE$\comment definition of $tr,start,end$\\
\axiom 2: \comment definition of $tr,start, end$\\
$\qquad\qquad csvfile  \in VALID\_FILE \implies$\\
$\qquad\qquad (\exists h,foot,s,e: \St; data:SEQ[ROW]$\\
$\qquad\qquad~~|~~~ h\in HEADER \cdot EV\_PER \cdot s \cdot``\_ \textrm{to}\_" \cdot e \cdot \textbf{*}``,"\cdot \textrm{eol}\cdot COL\_HEAD$\\
$\qquad\qquad~~~ \land data \in SEQ[ROW]$\\
$\qquad\qquad~~~ \land end \in \textbf{*}(\squote{,}|eol)$\\
$\qquad\qquad~~~ \land textfile \in h \cdot (\cdot i| 0 \leq i < \#data \itholds eol \cdot data(i)) \cdot foot$\\
$\qquad\qquad~ \itholds~~~ tr = (\cdot i | 0 \leq i < \#data \itholds <f(data(i))>$\\
$\qquad\qquad\qquad \land (start=s2d(s)) \land (end = s2d(e))$\\
$\qquad\qquad)$\\
}
\end{tikzpicture}
}
}
\caption{Type input-csv}
\label{fig:input}
\end{figure}

\newpage
%%%%%%%%%%%%%%%%%%%%%%%%%%%%%%%%%%%%%%%%%%%
%%%%%%%%%%%%%%%%%%%%%%%%%%%%%%%%%%%%%%%%%%%
%%%%%%%%%%%%%%%%%%%%%%%%%%%%%%%%%%%%%%%%%%%
%%%%%%%%%%%%%%%%%%%%%%%%%%%%%%%%%%%%%%%%%%%
\newcommand{\tab}{\hspace*{2em}}

{
\centering
\begin{longtable}{|l|}

\hline
\multicolumn{1}{|c|}{\textbf{TWR\_ROI\_CALCULATION}}\\

\hline
\textbf{\comment input} (\textit{input.csv})\\
\\
\textit{tr}: SEQ[TUPLE[\textit{date}: DATE, \textit{mv}: VALUE, \textit{cf}: VALUE, \textit{af}: VALUE, \\
\textit{bm}: VALUE $\bunion$ \{void\}]]\\
\comment sequence of transaction-tuples [\textit{date, market\_value, cash\_flow, agent\_fees,}\\
\comment \textit{benchmark}]\\
\comment \textit{tr.domain} = \{1,2,...,\textit{tr.count}\}\\
\\
\textit{count}: INTEGER$\eqdef$\textit{tr.count}\\
\\
\textit{dates}: SET[DATE]$\eqdef$\{\textit{t $\in$ tr$\itholds$t.date}\}\\
\\
\textit{start, end}: DATE$\comment$ metadata evaluation period\\
\\
\textit{duration}: VALUE$\eqdef$ \textit{days(end - start)}$\div$(365.2422) \\
\comment years between \textit{start} and \textit{end} calculated by days\\
\comment \textit{days}(x) similar to Excel\\

\\
\\
\textbf{\comment output calculation} (\textit{input.out.csv})\\
\\
\textit{di (d}:DATE): INTEGER\\
\comment index into sequence of transaction for date \textit{d}\\
\require \textit{d $\in$ dates} \\
\ensure \textit{Result $\in$ tr.domain $\land$ tr[Result].date=d}\\
\\

\comment TWR for the period \textit{start .. end}\\
\textit{twr (a\_start, a\_end}: DATE): VALUE\\
\require\\
	\tab \textit{a\_start, a\_end $\in$ dates}\\
	\tab \textit{a\_end $>$ a\_start}\\
	\tab \textit{$\forall$i $\in$ 2..count$\itholds$tr[i-1].mv + tr[i-1].cf + tr[i-1].af $\neq$ 0}\\
\ensure\\
	\tab \textit{Result $\eqdef$ ($\Pi$i:INTEGER $|$ di(a\_start) $<$i$\le$ di(a\_end)$\itholds$wealth(i)) - 1}\\
	\tab where \textit{wealth(i) $\eqdef$ tr(i].mv$\div$(tr[i-1].mv + tr[i-1].cf + tr[i-1].af)}\\
\\

\textit{annual\_compounded\_TWR (a\_start, a\_end}: DATE): VALUE \\
\ensure \\
	\tab \textit{(duration$\ge 1)\Rightarrow Result=((1+twr(a\_start, a\_end))^{1\div duration}-1)*100$}\\
	\tab \textit{(duration$<$1)$\Rightarrow$ Result= twr(a\_start, a\_end) * 100}\\
\\

\textit{roi (a\_start, a\_end}: DATE): VALUE\\
\require\\
	\tab \textit{a\_start, a\_end $\in$ dates}\\
	\tab\textit{a\_end $>$ a\_start}\\
\ensure\\
	\tab \textit{(tr[m].mv+tr[m].cf) $\ast (1+Result\div 100)^{days(a\_end - a\_start)\div365.2422}$ }\\
	\tab \tab \textit{+ ($\Sigma i:\nat  |  m <i<n\itholds (tr[i].cf + tr[i].af) \ast $}\\
	\tab \tab \textit{$(1+Result\div 100)^{days(a\_end - tr[i].date)\div365.2422}) - tr[n].mv = 0$}\\
	\tab where \textit{m = di(a\_start)}\\ 
	\tab \tab \hspace*{1em} \textit{n = di(a\_end)}\\
\\

\textit{benchmark\_calculable (a\_start, a\_end}: DATE): $\Bool$\\
\require\\
	\tab \textit{\{$\forall t \in tr  |  t.date="yyyy-01-01" \itholds t.bench \neq void$\}}\\
	\tab \textit {$\bunion \{tr[di(a\_end)].bench \neq void\}$}\\
\ensure\\
	\tab \textit {Result = $\True$} \\
\\

\comment the function below return a set of index corresponding to the date with\\
\comment a benchmark\\
\textit{bench\_seq (a\_start, a\_end}: DATE): SEQ[INTEGER]\\
\require\\
	\tab \textit{benchmark\_calculable(a\_start, a\_end) }\\
\ensure\\
	\tab \textit{$\{\forall i \in Result \itholds tr[i].bench \neq void \wedge (di(a\_start)<i\le di(a\_end))\}$}\\
	\tab \comment \textit{ Result.domain = \{1,2..Result.count\}}\\
\\

\textit{benchmark (a\_start, a\_end}: DATE): VALUE\\
\require\\
	\tab \textit{benchmark\_calculable(a\_start, a\_end) }\\
\ensure\\

	\tab \textit{$(tr[m].mv+tr[m].cf) \ast (Result+1)^{days(a\_end - a\_start)\div365.2422} $}\\
	\tab \textit{$+ (\Sigma k: VALUE | m+1 \le k \le n \itholds tr[k].cf \ast$}\\
	\tab \tab \textit{$(Result+1)^{days(a\_end - tr[k].date)\div365.2422} = FV$}\\
\\

	\tab where \textit{m = di(a\_start)}\\ 
	\tab \tab \hspace*{1em} \textit{n = di(a\_end)}\\
	\tab \tab \hspace*{1em} \textit{$FV \eqdef  $}\\
	\tab \textit{$(tr[m].mv+tr[m].cf) \ast (\Pi i: VALUE | s = bench\_seq(a\_start, a\_end) $} \\
	\tab \tab \textit{$ \wedge s[0] = m\wedge i \in s \wedge i \ge 1 \wedge i = s(j) \itholds $}\\
 	\tab \tab \textit{$tr[i].bench^{days(tr[i].date - tr[s(j-1)].date)\div365.2422} )$}\\
	\tab \textit{$+ (\Sigma k: VALUE | m+1 \le k \le n \itholds (tr[k].cf - tr[k].af) \ast$}\\
	\tab \tab \textit{$(\Pi i: VALUE | s = bench\_seq(tr[k].date, a\_end) \wedge s[0] = k \wedge i \in s$}\\
	\tab \tab \textit{$\wedge  i \ge 1 \wedge i = s(j) \itholds tr[k].bench^{days(tr[i].date-tr[s(j-1)].date)\div365.2422}))$}\\
\\


\hline
\textbf{Invariants}\\
\\
(1) \textit{ (start$<$end)$\wedge$(start,end$\in$dates)} \\
\comment metadata evaluation period is in range and valid\\
\\
(2) \textit{$\forall$t$\in$tr$\itholds$t.date$\neq$Void$\wedge$t.mv$\ge$0}\\
\comment every row has a date and a non-negative market value\\
\\
(3) \textit{$\forall$i$\in$2..count$\itholds$tr[i].date$>$tr[i-1].date}\\
\comment date are unique and ordered\\
\\
(4) \textit{$\forall$t$\in$tr$\itholds$t.mv+t.cf$\ge$0}\\
\comment Cannot withdraw more than the market value\\
\\
(5) \textit{$\forall$i$\in$2..count $|$ tr[i-1].mv=0$\wedge$tr[i-1].cf=0$\itholds$tr[i].mv=0}\\
\comment account coannot grow from zero market value and cash flow\\

\hline
\caption{Mathematical model for the ROI system}
\label{tab:twr_calculation}
\end{longtable}
}


%%%%%%%%%%%%%%%%%%%%%%%%%%%%%%%%%%%%%%%%%%%
%%%%%%%%%%%%%%%%%%%%%%%%%%%%%%%%%%%%%%%%%%%
%%%%%%%%%%%%%%%%%%%%%%%%%%%%%%%%%%%%%%%%%%%
%%%%%%%%%%%%%%%%%%%%%%%%%%%%%%%%%%%%%%%%%%%



%%%%%%%%%%%%%%%%%%%%%%%%%%%%%%%%%%%%%%%%%

\bibliographystyle{plain}
\bibliography{inputs/ref}







\newpage
\begin{landscape}
\begin{table}

\centering

\begin{tabular}{|l|l|l|l|l|l||l|l|l|l|l|}
\hline
\multicolumn{6}{|c||}{} & $Error$ & $Warning$ & \multicolumn{3}{|c|}{Whole input}\\ 


\multicolumn{6}{|c||}{} & & & TWR & ROI & B\\ 


\hline
Valid$\_$CSV & \multicolumn{5}{|l||}{Start$\_$Invalid $\vee$ } & E1 & | & | & | & |\\ 
& \multicolumn{5}{|l||}{End$\_$Invalid $\vee$}& & & &&\\
& \multicolumn{5}{|l||}{end $\le$ start}& & & &&\\

\cline{2-11}
& Start$\_$Valid  $\wedge$ & C1 & \multicolumn{3}{|l||}{C2} & | & | & a$\_$c$\_$TWR(all) & roi(all) & b(all)\\
& End$\_$Valid $\wedge$& &\multicolumn{3}{|l||}{}&&&&&\\
& end$>$start &&\multicolumn{3}{|l||}{}&&&&&\\

\cline{4-11}
& & & $\lnot$C2 &\multicolumn{2}{|l||}{C4} & | & W1 & a$\_$c$\_$TWR(all) & roi(all) & |\\

\cline{5-11}
& & & &\multicolumn{2}{|l||}{$\lnot$C4} & | & W2 & a$\_$c$\_$TWR(all) & roi(all) & |\\

\cline{3-11}
& & $\lnot$C1 & C3 &\multicolumn{2}{|l||}{C2} & | & W3 & | & roi(all) & b(all)\\

\cline{5-11}
& & & & $\lnot$C2 & C4 & | & W4 & | & roi(all) & |\\

\cline{6-11}
& & & & & $\lnot$C4 & | & W5 & | & roi(all) & |\\

\cline{4-11}
& & & $\lnot$C3 &\multicolumn{2}{|l||}{C2} & | & W6 & | & roi(all) & b(all)\\

\cline{5-11}
& & & & $\lnot$C2 & C4 & | & W7 & | & roi(all) & |\\

\cline{6-11}
& & & & & $\lnot$C4 & | & W8 & | & roi(all) & |\\

\hline
\multicolumn{6}{|l||}{Invalid$\_$CSV} & E2 & | & | & | & |\\ 


\hline
\end{tabular}
\caption{Function table for ROI system (error, warning and whole input)}
\label{table:table_ROI}
\end{table}
\end{landscape}




\newpage
\begin{landscape}
\begin{table}
\centering

\begin{tabular}{|l|l|l|l|l|l||l|l|l|}
\hline
\multicolumn{6}{|c||}{} & \multicolumn{3}{|c|}{Evaluation period}\\ 

\multicolumn{6}{|c||}{} & TWR & ROI & B\\ 


\hline
Valid$\_$CSV & \multicolumn{5}{|l||}{Start$\_$Invalid $\vee$ } & | & | & |\\ 
& \multicolumn{5}{|l||}{End$\_$Invalid $\vee$} & &&\\
& \multicolumn{5}{|l||}{end $\le$ start}& & &\\

\cline{2-9}
& Start$\_$Valid  $\wedge$ & C1 & \multicolumn{3}{|l||}{C2} & a$\_$c$\_$TWR(start,end) & roi(start,end) & b(start,end)\\
& End$\_$Valid $\wedge$& &\multicolumn{3}{|l||}{}&&&\\
& end$>$start &&\multicolumn{3}{|l||}{}&&&\\

\cline{4-9}
& & & $\lnot$C2 &\multicolumn{2}{|l||}{C4} & a$\_$c$\_$TWR(start,end) & roi(start,end) & b(start,end)\\

\cline{5-9}
& & & &\multicolumn{2}{|l||}{$\lnot$C4} & a$\_$c$\_$TWR(start,end) & roi(start,end) & |\\

\cline{3-9}
& & $\lnot$C1 & C3 &\multicolumn{2}{|l||}{C2} & a$\_$c$\_$TWR(start,end) & roi(start,end) & b(start,end)\\

\cline{5-9}
& & & & $\lnot$C2 & C4 & a$\_$c$\_$TWR(start,end) & roi(start,end) & b(start,end)\\

\cline{6-9}
& & & & & $\lnot$C4 & a$\_$c$\_$TWR(start,end) & roi(start,end) & |\\

\cline{4-9}
& & & $\lnot$C3 &\multicolumn{2}{|l||}{C2} & | & roi(start,end) & b(start,end)\\

\cline{5-9}
& & & & $\lnot$C2 & C4 & | & roi(start,end) & b(start,end)\\

\cline{6-9}
& & & & & $\lnot$C4 & | & roi(start,end) & |\\


\hline
\multicolumn{6}{|l||}{Invalid$\_$CSV} & | & | & |\\ 


\hline
\end{tabular}




\caption{Function table for ROI system (evaluation period)}
\label{table:table_ROI_2}
\end{table}
\end{landscape}


\bigskip
The function tables use some abbreviations:\\
\begin{itemize}
\item a\_c\_TWR = annual\_compounded\_TWR (see TWR\_ROI\_CALCULATION (Fig.~\ref{tab:twr_calculation}))\\
\item b = benchmark (see TWR\_ROI\_CALCULATION (Fig.~\ref{tab:twr_calculation}))\\
\item function(all) = function(tr[1].date, tr[tr.count].date)\\
\end{itemize}
\bigskip
\bigskip
The function tables use also conditions:\\
\begin{itemize}
\item C1 = \textit{$\forall$i $\in$ 2..count$\itholds$tr[i-1].mv + tr[i-1].cf + tr[i-1].af $\neq$ 0}\\
\item C2 = benchmark\_calculable(all) (see TWR\_ROI\_CALCULATION (Fig.~\ref{tab:twr_calculation}))\\
\item C3 = \textit{$\forall$i $\in$ di(start)+1..di(end)$\itholds$tr[i-1].mv + tr[i-1].cf + tr[i-1].af $\neq$ 0}\\
\item C4 = benchmark\_calculable(start,end) (see TWR\_ROI\_CALCULATION (Fig.~\ref{tab:twr_calculation}))\\
\item Start\_Valid = $\lnot$ (Start\_Invalid) = (start $\in$ dates$\cup$\{null\}) $\wedge$ (start in ISO format)\\
(if Start\_Valid $\wedge$ start=null, then start=tr[1].date)\\
\item End\_Valid = $\lnot$ (End\_Invalid) = (end $\in$ dates$\cup$\{null\}) $\wedge$ (end in ISO format)\\
(if End\_Valid $\wedge$ end=null, then end=tr[tr.count].date)\\


\end{itemize}
\bigskip
\bigskip
The function tables provide errors and warnings messages:\\
\begin{itemize}
\item E1 = "Invalid\_Evaluation\_Period"
\item E2 = "Invalid\_file"
\item W1 =  "Benchmark for the whole input is not calculable"
\item W2 = "The benchmarks are not calculable"
\item W3 = "The TWR for the whole input is not calculable"
\item W4 = W1+W3
\item W5 = W1+W2+W3
\item W6 = "The TWR's are not calculable"
\item W7 = W1+W6
\item W8 = W2+W6
\end{itemize}



\newpage
\appendix

\section{REGEXP}

\begin{figure}
\resizebox{\textwidth}{!}
{
\begin{tikzpicture}
\renewcommand\tikzumlfillclasscolor{white}
\umlclass{type REGEXP}
{
\cset $\regexp$\comment set of all regular string expressions\\
\axiom $\regexp \subseteq \pow(\St)$\\
\cset $\Sigma \eqdef \{``0", ``1",  ``2", \cdots, ``a", ``b",  \textrm{ etc., all printing characters} \}$\\
}
{

\dummy $x,y,z : \regexp$\\
\dummy $s,t,u : \St$\\

\axiom $\forall s\in\Sigma \itholds \{s\} \in \regexp$\\

\const $0:\regexp \eqdef \{\}$\comment zero is the unit element of alternation\\
\const $1:\regexp\eqdef \{``"\}$\comment $1$ is the unit element of concatenation\\
\comment we also use $\epsilon$ instead of $1$\\

\const \infix $``|": \regexp \cprod \regexp \tfun \regexp$\\
\comment alternation\\
\const \infix $``\cdot": \regexp \cprod \regexp \tfun \regexp$\\
\comment concatenation\\

\const \prefix$``*": \regexp \cprod \regexp \tfun \regexp$\\
\comment iteration zero or more times\\
\const \prefix$``\textrm{+}": \regexp \cprod \regexp \tfun \regexp$\\
\comment iteration one or more times\\

\axiom $s \in x|y ~\equiv~ s\in x \,\lor\, s\in y$\\
\theorem $x | 0 = 0 | x = x$\\
\axiom $s \in x\cdot y ~\equiv~ (\exists t,u| s=t\cdot u \itholds t\in x \land u\in y)$\\
\comment note that $t\cdot u$ is concatenation over $SEQ[\St]$\\
\theorem $1 \cdot x ~=~ x \cdot 1 ~=~ 1$\
\comment 1 is the identity of concatenation\\

\const \infix ``\^{}"$: \regexp \cprod \nat \tfun \regexp$\\
\comment use this operator by raising the second argument like an exponent\\
\axiom $x^n = (\cdot i\suchthat 0 \leq i \leq n \itholds x)$\comment concatenation quantifier\\

\comment e.g. $x^3 = x \cdot x \cdot x$\\
\theorem $x^0 = 1$\\
\axiom $s \in *x ~\equiv~ (\exists n : \nat \itholds s \in x^n)$\\
\axiom $s\in \textrm{+}x \equiv (\exists n : \natn \itholds s \in x^n)$
}
\end{tikzpicture}
}



%  x^0
%=    { definition }
%   (⊙i : 0 ≤ i < 0 : x)
%=    { arithmetic }
%   (⊙i : false : x)
%=    { identity of ∙ }
%   1

% w ∈ x∙1
%=    { definition of ∙ }
%   (∃u,t:  w = u∙t:  u ∈ x ∧ t ∈ 1)
%=    { definition of 1 }
%   (∃u,t:  w = u∙t:  u ∈ x ∧ t ∈ { "" })
%=    { membership of a singleton set }
%   (∃u,t:  w = u∙t:  u ∈ x ∧ t = "")
%=    { one point rule }
%   (∃u:  w = u∙"":  u ∈ x)
%=    { identity of string catenation }
%   (∃u:  w = u:  u ∈ x)
%=    { one point rule }
%   w ∈ x
%%%%%%%%%%%%%%%%%%
\caption{Type REGEXP for regular expressions over printing characters}
\end{figure}

A set of strings is used as the model for regular expressions. We use prefix operators for the Kleene closure (e.g. $*x$ where $x$ is a regular expression such as $\{\squote{hello}\}$) and iteration at least one or more (e.g. $\textrm{+}x$) rather than suffix operators. Note that where there is no confusion we use \squote{hello} instead of $\{\squote{hello}\}$ where the set is a singleton. 

We may use type REGEXP to specify a $FLOAT\_STRING$ as follows.

\begin{align}
FLOAT\_STRING = &\squote{+}Inf\\
&|\squote{-}Inf\\
&|NaN\\
&|(\squote{-}|\squote{+}|\epsilon)\cdot(^*d\cdot\squote{.}|\epsilon)\cdot
	^*\!\!d\cdot ((\squote{e}\cdot(\squote{-}|\epsilon)\cdot^\textrm{+}d)\,|\,\epsilon)\\
d = &\squote{0} | \squote{1} | \cdots |\squote{9} 
\end{align}

In the above we use the convention that \squote{e}, for example, really stands for the single set $\{\squote{e}\}$.

%  x^0
%=    { definition }
%   (⊙i : 0 ≤ i < 0 : x)
%=    { arithmetic }
%   (⊙i : false : x)
%=    { identity of ∙ }
%   1

% w ∈ x∙1
%=    { definition of ∙ }
%   (∃u,t:  w = u∙t:  u ∈ x ∧ t ∈ 1)
%=    { definition of 1 }
%   (∃u,t:  w = u∙t:  u ∈ x ∧ t ∈ { "" })
%=    { membership of a singleton set }
%   (∃u,t:  w = u∙t:  u ∈ x ∧ t = "")
%=    { one point rule }
%   (∃u:  w = u∙"":  u ∈ x)
%=    { identity of string catenation }
%   (∃u:  w = u:  u ∈ x)
%=    { one point rule }
%   w ∈ x
%%%%%%%%%%%%%%%%%%



\end{document}



\caption{Excel CSV input file}
\label{fig:csv}
\end{figure}

Customers do not want to pay a lot of money for the software and so they are prepared to forgo many things --- a minimalistic product is expected. The product may be used via a command line interface (it may also have a simple GUI or can be mounted as a properly secured web application). 


\section{Context Diagram}

{\color{red} Provide a context diagram with precise description of monitored and controlled variables. Indicate the entities in the environment. Note that in the sequel below we provide the precise nature of the input. You must elicit the precise outputs that are required.}

\section{Dictionary}

The dictionary is incomplete

\smallskip

\noindent\textbf{CSV}: Comma Separated Value file format used to store tabular data in which numbers and text are stored in plain-text form that can be easily written and read in a text editor.\smallskip

\noindent\textbf{Customer}: The user of the software system.\smallskip

\noindent\textbf{Evaluation Period}: a start and end date (provided by the user) for the portfolio history over which the return on investment is calculated.\smallskip

\noindent\textbf{GIPS}: Global Investment Performance Standards [1]\smallskip

\noindent\textbf{Investment broker}: Runs the portfolio on behalf of the customer and supplies portfolio accounts.\smallskip

\noindent\textbf{Portfolio statement}: List of all investments and current value.\smallskip

\noindent\textbf{Portfolio History}: the historical data of investment performance over time that the customer stores about their investments as gleaned from their monthly or yearly investment accounts. Uusally stored by customers in a CSV file (see Figure 1).\smallskip

\noindent\textbf{TWR}: Time Weighted Return (see [1])\smallskip

%%%%%%%%%%%%%%%%%%%%%%%%%%%%%%%%%%%
\section{E/R-descriptions}
Fill this in ..

\reqm{ENV}
{Description\\}
{References}

\reqm{REQ}
{Description\\}
{References}

{\color{red} Note that you must calculate the compounded TWR (and its annualized value) for the complete period as well as for the evaluation period. The TWR is not always accurate. You must provide an accurate caculation (called precise-ROI).}

%%%%%%%%%%%%%%%%%%%%%%%%%%%%%%%%%%%
\section{Mathematical model}
%%%%%%%%%%%%%%%%%%%%%%%%%%%%%%%%%%%

{\color{red} We provide below an incomplete mathematical model for ROI system. We define a valid input CSV file as a regular expression. Obviously there must be an error report for files that do not match the precise specification of input. We also provide the outline of a type to calculate the TWR and precise-ROI, which you must complete. You will also need a function table to ensure that all possible inputs (including faulty inputs) are handled.}

\begin{figure}
\centering
\includegraphics[scale=0.8]{inputs/spec.pdf}
%\begin{figure}
\resizebox{\textwidth}{!}
{
\def\data{type DATA}
\def\input{type input-csv}
\def\roi{type roi}
\def\control{module control}
\renewcommand\tikzumlfillclasscolor{white}
\def\do{\quad\textbf{do}~} %%Needed
\centering
\resizebox{\textwidth}{!}{\begin{tikzpicture}

\umlclass{\data}
{
\use DATE\comment see birthday book\\
\cset $DATA \eqdef$\\
$\qquad  Data(date: DATE, mv:\Fl, cf:\Fl, af:\Fl,bench: \Fl\bunion\{Void\})$\\
$\qquad$\require $valid\_date(date)$\\

}
{
\comment the above declaration induces an injective function with accessors\\
$Data: DATE \cprod \Fl \cprod \Fl \cprod \Fl \cprod \Fl\bunion\{Void\} \tinj DATA$\\
\const $mv,cf,af : DATA \tfun \Fl$\\
\comment market-value, cash-flow and agent-fees\\
\const $bench: DATA \tfun \Fl\bunion\{Void\}$\comment benchmark\\
\axiom $\forall d:DATA \itholds d=Data(date(d),mv(d),cf(d),bench(d))$\\
\comment e.g. $DATA = \{ Data(\squote{2007-02-11},105000,15000,0,Void), \cdots\}$
}
        
\umlclass[x=-4.5, y=-6.4]{\input}
{
\use type $REGEXP, DATA$ \\
\textbf{param} $csvfile : \St$\\
\comment input parameter\\
\query $tr : SEQ[DATA]$\\
\comment sequence of transactions\\
\require $csvfile  \in VALID\_FILE$\\
$\query start, end: DATE$\\
\require $csvfile  \in VALID\_FILE$\\
}
{
\comment see Fig.~\ref{fig:input} for the details
}
        
\umlclass[x=4.5, y=-6.4]{\roi}
{
\use $input(cvsfile)$\\
\param $cvsfile:\St$\\
\query $report:\St$
}
{
\comment see function table for outputs\\
}
        
\umlclass[y=-12.4]{\control}
{
\comment specification pattern for transfomational programs
}
{
\event $roiReport \leftarrow invoke$\\
\any $file:\St$\\
$\quad$\use $roi(file)$ \textbf{as} $r$\\
\ddo $roiReport := r.report$\\\\
\comment dynamic use of module \textbf{roi}\\
\comment instantiate $cvsfile$ as $file$

}
        
\umlunicompo{\input}{\data}
\umlunicompo{\roi}{\data}
\umlunicompo{\roi}{\input}
\umlunicompo{\control}{\roi}
\end{tikzpicture}}
}
%\end{figure}
\caption{Module specification of return on investment}
\end{figure}

\begin{figure}
\resizebox{\textwidth}{!}{{\begin{tikzpicture}
\renewcommand\tikzumlfillclasscolor{white}
\umlclass{type input-csv}
{
\use type $REGEXP, DATA, DATE$ 
\comment we let $\epsilon = \{``"\}$, eol = $\{``\backslash\textrm{n}"\}$ etc.\\
\cset $DATA \eqdef Data(date: DATE, mv:\Fl, cf:\Fl, af:\Fl,bench: \Fl\bunion\{Void\})$\\
\textbf{param} $csvfile : \St$\comment input parameter\\
\query $tr : SEQ[DATA]$\comment sequence of transactions defined by axiom below\\
\require $csvfile  \in VALID\_FILE$\\
$\query start, end: DATE$\\
\require $csvfile  \in VALID\_FILE$\\
}
{
\const $VALID\_FILE : \regexp$\\
$\eqdef HEADER \cdot PARAMETERS \cdot \textrm{eol} \cdot ROW \cdot \textbf{*}(\textrm{eol} \cdot ROW)\cdot \textbf{*}(``,"|\textrm{eol})$\\
\const $HEADER: REGEXP$\\
$\eqdef \textbf{*} (HLINE \cdot eol) $\\
\const $HLINE:REGEXP$\\
$\eqdef \textbf{*} (\Sigma \backslash \textrm{eol}) \backslash (EV\_PER\cdot \textbf{*} \Sigma)$\\
\const $PARAMETERS:REGEXP$\\
$\eqdef EV\_PER \cdot DATE\_STR \cdot``\_ \textrm{to}\_" \cdot DATE\_STR \cdot \textbf{*}``,"\cdot \textrm{eol}\cdot COL\_HEAD$\\
\const $COL\_HEAD:REGEXP$\\
$ \eqdef +``," \cdot eol \cdot$\\ $\qquad``\textrm{Transaction\_Date,Market\_Value,Cash\_Flow,Agent\_Fees,Benchmark}"\cdot\textbf{*} ``,"$\\
\const $EV\_PER:REGEXP \eqdef ``\textrm{Evaluation\_Period:\_}" $\\
\const $ROW: \regexp$\\
$\eqdef (DATE\_STR \cdot ``,"\cdot  FLOAT\cdot ``,"  \cdot (FLOAT|\epsilon) \cdot ``," \cdot (FLOAT|\epsilon)\cdot$\\
$\qquad ``,"  \cdot (FLOAT\cdot``\%"|\epsilon)\cdot \textbf{*}``,")$\\
\const $s2d : DATE\_STR \tfun DATE$\comment see birthday book for $DATE$\\
\const $s2f: FLOAT \tfun \Fl$\comment deferred, $FLOAT$ is the string version of $\Fl$\\
\const $f2s : \Fl \tfun FLOAT$\comment deferred, see your favourite programming language\\
\const $d2s : DATE \tfun DATE\_STR$\comment deferred\\
\const $s2optf[G]: (FLOAT | \epsilon)\cprod G\tfun  \Fl \bunion G$\comment string-to-optional float \\
\where $\forall G \itholds s2optf \in (FLOAT | \epsilon)\cprod G\tfun  \Fl \bunion G$\\
\comment parameter $G$ is a set such as $\{Void\}$ or a default value such as $\{0\}$\\
\const $f : ROW \tfun DATA$\\
\textbf{dummy} $w:ROW$ and $s_0, s_1, s_2,s_3:\St$\\
\axiom 1:\comment definition of function $f$ that maps a row string to data\\
$\qquad\qquad w \in  (d2s(d)\cdot ``,"\cdot s_0 \cdot ``,"\cdot s_1 \cdot ``,"\cdot s_2 \cdot ``," \cdot s_3 \cdot \textbf{*}``,") $ \\ 
$\qquad ~~\land ~ (s_4 \cdot ``\%" = s_3 \lor s_4 = s_3 = \epsilon)$ \\
$\qquad \implies   f (w) = Data(d, s2f(s_0), s2optf(s_1,0), s2optf(s_2,0), s2optf(s_4,\text{Void}))$\\
\query $error : \Bl \eqdef textfile \not\in VALID\_FILE$\comment definition of $tr,start,end$\\
\axiom 2: \comment definition of $tr,start, end$\\
$\qquad\qquad csvfile  \in VALID\_FILE \implies$\\
$\qquad\qquad (\exists h,foot,s,e: \St; data:SEQ[ROW]$\\
$\qquad\qquad~~|~~~ h\in HEADER \cdot EV\_PER \cdot s \cdot``\_ \textrm{to}\_" \cdot e \cdot \textbf{*}``,"\cdot \textrm{eol}\cdot COL\_HEAD$\\
$\qquad\qquad~~~ \land data \in SEQ[ROW]$\\
$\qquad\qquad~~~ \land end \in \textbf{*}(\squote{,}|eol)$\\
$\qquad\qquad~~~ \land textfile \in h \cdot (\cdot i| 0 \leq i < \#data \itholds eol \cdot data(i)) \cdot foot$\\
$\qquad\qquad~ \itholds~~~ tr = (\cdot i | 0 \leq i < \#data \itholds <f(data(i))>$\\
$\qquad\qquad\qquad \land (start=s2d(s)) \land (end = s2d(e))$\\
$\qquad\qquad)$\\
}
\end{tikzpicture}
}
}
\caption{Type csv-input}
\label{fig:input}
\end{figure}


	


\section{Acceptance Tests}

\noindent {\color{red} Very incomplete. You need a large number of tests including error tests}\bigskip

\noindent
\begin{tabular}{|p{1in}|p{4in}|}
\hline
Test Case ID	 & T1 \\ 
\hline
Description & Verify that return on investment (compounded TWR) is calculated correctly\\
\hline
Requirement IDs tested & R1? \\ 
\hline
Type & Positive \\ 
\hline
Initial State & A directory containing the CSV file in Figure~\ref{fig:csv} \\
\hline 
Action & Execute the ROI system on the CSV file \\
\hline 
Consequences & The ROI system reports the compounded TWR as 144.08\% \\ 
\hline
\end{tabular}

\bigskip
\noindent
\begin{tabular}{|p{1in}|p{4in}|}
\hline
Test Case ID	 & T2 \\ 
\hline
Description & Verify that return on investment (compounded TWR) is calculated correctly\\
\hline
Requirement IDs tested & R1? \\ 
\hline
Type & Positive \\ 
\hline
Initial State & A directory containing the CSV file in Figure~\ref{fig:csv} 
with the evaluation period from 2007-01-01 to 2009-04-01\\
\hline 
Action & Execute the ROI system on the CSV file \\
\hline 
Consequences & The ROI system reports the compounded TWR as 82.49\% \\ 
\hline
\end{tabular}

\section{Requirements Traceability matrix}

\noindent
\begin{tabular}{|p{1.5in}|p{3.5in}|}
\hline
\textbf{Requirement ID}	 & \textbf{Test Case IDs}\\ 
\hline
R1 & T1, T2, ...\\
\hline
R2 & ...\\ 
\hline
R3 & ...\\ 
\hline

\end{tabular}

%%%%%%%%%%%%%%%%%%%%%%%%%%%%%%%%%%%%%%%%%

\bibliographystyle{plain}
\bibliography{inputs/ref}

\appendix

\section{REGEXP}

\begin{figure}
\resizebox{\textwidth}{!}
{
\begin{tikzpicture}
\renewcommand\tikzumlfillclasscolor{white}
\umlclass{type REGEXP}
{
\cset $\regexp$\comment set of all regular string expressions\\
\axiom $\regexp \subseteq \pow(\St)$\\
\cset $\Sigma \eqdef \{``0", ``1",  ``2", \cdots, ``a", ``b",  \textrm{ etc., all printing characters} \}$\\
}
{

\dummy $x,y,z : \regexp$\\
\dummy $s,t,u : \St$\\

\axiom $\forall s\in\Sigma \itholds \{s\} \in \regexp$\\

\const $0:\regexp \eqdef \{\}$\comment zero is the unit element of alternation\\
\const $1:\regexp\eqdef \{``"\}$\comment $1$ is the unit element of concatenation\\
\comment we also use $\epsilon$ instead of $1$\\

\const \infix $``|": \regexp \cprod \regexp \tfun \regexp$\\
\comment alternation\\
\const \infix $``\cdot": \regexp \cprod \regexp \tfun \regexp$\\
\comment concatenation\\

\const \prefix$``*": \regexp \cprod \regexp \tfun \regexp$\\
\comment iteration zero or more times\\
\const \prefix$``\textrm{+}": \regexp \cprod \regexp \tfun \regexp$\\
\comment iteration one or more times\\

\axiom $s \in x|y ~\equiv~ s\in x \,\lor\, s\in y$\\
\theorem $x | 0 = 0 | x = x$\\
\axiom $s \in x\cdot y ~\equiv~ (\exists t,u| s=t\cdot u \itholds t\in x \land u\in y)$\\
\comment note that $t\cdot u$ is concatenation over $SEQ[\St]$\\
\theorem $1 \cdot x ~=~ x \cdot 1 ~=~ 1$\
\comment 1 is the identity of concatenation\\

\const \infix ``\^{}"$: \regexp \cprod \nat \tfun \regexp$\\
\comment use this operator by raising the second argument like an exponent\\
\axiom $x^n = (\cdot i\suchthat 0 \leq i \leq n \itholds x)$\comment concatenation quantifier\\

\comment e.g. $x^3 = x \cdot x \cdot x$\\
\theorem $x^0 = 1$\\
\axiom $s \in *x ~\equiv~ (\exists n : \nat \itholds s \in x^n)$\\
\axiom $s\in \textrm{+}x \equiv (\exists n : \natn \itholds s \in x^n)$
}
\end{tikzpicture}
}



%  x^0
%=    { definition }
%   (⊙i : 0 ≤ i < 0 : x)
%=    { arithmetic }
%   (⊙i : false : x)
%=    { identity of ∙ }
%   1

% w ∈ x∙1
%=    { definition of ∙ }
%   (∃u,t:  w = u∙t:  u ∈ x ∧ t ∈ 1)
%=    { definition of 1 }
%   (∃u,t:  w = u∙t:  u ∈ x ∧ t ∈ { "" })
%=    { membership of a singleton set }
%   (∃u,t:  w = u∙t:  u ∈ x ∧ t = "")
%=    { one point rule }
%   (∃u:  w = u∙"":  u ∈ x)
%=    { identity of string catenation }
%   (∃u:  w = u:  u ∈ x)
%=    { one point rule }
%   w ∈ x
%%%%%%%%%%%%%%%%%%
\caption{Type REGEXP for regular expressions over printing characters}
\end{figure}

A set of strings is used as the model for regular expressions. We use prefix operators for the Kleene closure (e.g. $*x$ where $x$ is a regular expression such as $\{\squote{hello}\}$) and iteration at least one or more (e.g. $\textrm{+}x$) rather than suffix operators. Note that where there is no confusion we use \squote{hello} instead of $\{\squote{hello}\}$ where the set is a singleton. 

We may use type REGEXP to specify a $FLOAT\_STRING$ as follows.

\begin{align}
FLOAT\_STRING = &\squote{+}Inf\\
&|\squote{-}Inf\\
&|NaN\\
&|(\squote{-}|\squote{+}|\epsilon)\cdot(^*d\cdot\squote{.}|\epsilon)\cdot
	^*\!\!d\cdot ((\squote{e}\cdot(\squote{-}|\epsilon)\cdot^\textrm{+}d)\,|\,\epsilon)\\
d = &\squote{0} | \squote{1} | \cdots |\squote{9} 
\end{align}

In the above we use the convention that \squote{e}, for example, really stands for the single set $\{\squote{e}\}$.

%  x^0
%=    { definition }
%   (⊙i : 0 ≤ i < 0 : x)
%=    { arithmetic }
%   (⊙i : false : x)
%=    { identity of ∙ }
%   1

% w ∈ x∙1
%=    { definition of ∙ }
%   (∃u,t:  w = u∙t:  u ∈ x ∧ t ∈ 1)
%=    { definition of 1 }
%   (∃u,t:  w = u∙t:  u ∈ x ∧ t ∈ { "" })
%=    { membership of a singleton set }
%   (∃u,t:  w = u∙t:  u ∈ x ∧ t = "")
%=    { one point rule }
%   (∃u:  w = u∙"":  u ∈ x)
%=    { identity of string catenation }
%   (∃u:  w = u:  u ∈ x)
%=    { one point rule }
%   w ∈ x
%%%%%%%%%%%%%%%%%%

\section{Precise calculation of ROI}

The TWR is only an approximation to the real time-weighted return, as in  Fig.~\ref{fig:twr} (where one can see that the infusion of \$500 in cash reduces the ROI). 

Fig.~\ref{fig:bad-twr} shows where the approximation goes badly wrong. In this case, the investment advisor made a huge profit for our client and the gain is 16.19\%, whereas the TWR shows the advisor as making a loss of 5.58\%.

Clearly, a more precise method is called for. 



\begin{figure}[!]
\centering
\includegraphics[scale=0.8]{inputs/twr.pdf}
\caption{TWR as a good approximation}
\label{fig:twr}
\end{figure}

\begin{figure}[!]
\centering
\includegraphics[scale=2.3]{inputs/bad-twr.png}
\caption{TWR as a bad approximation}
\label{fig:bad-twr}
\end{figure}

\subsection{Compound interest}
Suppose you invest \$1000 for 5 years at 10\% per annum. So we know that after the first year we have $1000*R = 1000*1.10 = 1100$ (where $R$ is the rate of return multiplier, i.e. $R=1.1$). For 5 years we have $1000*R^5 = 1610.51$. The general formula is

\[ PV * (1+r)^n = FV\]

\noindent where $r$ is the interest rate as a decimal (e.g. 0.1, i.e. 10\%), and $n$ is the number of periods. If $R=1+r$ then we have $PV * R^n = FV$


For the second example in Fig.~\ref{fig:twr}, it is not so simple as we are not adding amounts at regular intervals (cash flow in/out is irregular). We may use $n$ as a day (i.e. $1/365$ of a year) and we get:

\[
FV = (PV *R^{365/365}) + (500*R^{116/365})
\]

i.e. we weight the initial money ($PV = 10000$) by the full year (365 days) of daily return. The input of cash in Septmember (which is 116 days to the end of the years) is weighted in that proportion. The equation van be re-written

\[
(PV *R^{365/365}) + (500*R^{116/365}) -FV = 0
\]

and we can solve for the root of the polynomial (e.g. using Newton-Raphson) to obtain $R$. This gives us a net gain of $4.923$\%. The TWR calculated it as $5$\%, which is only approximately correct.

The general formulas is
\begin{align}
\small
& tr(1).mv * R^{\Delta days(i)} + \\\nonumber
& \qquad (\Sigma i: \nat|1 < i < n\itholds tr(i).cf * R^{\Delta days(i)})- tr(n).mv = 0\\
& \textbf{where } n = \card(tr)& \nonumber \\
& end = tr(n).date&\\\nonumber
&\Delta days(i) = (end- tr(i).date)/365&\nonumber
\end{align}

\end{document}


